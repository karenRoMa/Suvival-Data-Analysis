%%% LaTeX Template: Article/Thesis/etc. with colored headings and special fonts
%%%
%%% Source: http://www.howtotex.com/
%%% Feel free to distribute this template, but please keep to referal to http://www.howtotex.com/ here.
%%% February 2011

%%%%% Preamble
\documentclass[10pt,a4paper]{article}
\usepackage[utf8]{inputenc}
\usepackage[T1]{fontenc}
\usepackage[bitstream-charter]{mathdesign}						% Input encoding
\usepackage{amsmath}									% Math

\usepackage{xcolor}
\definecolor{bl}{rgb}{0.529,0.149,0.341}

\usepackage{sectsty}
\usepackage[compact]{titlesec} 
\allsectionsfont{\color{bl}\scshape\selectfont}

%%%%% Definitions
% Define a new command that prints the title only
\makeatletter							% Begin definition
\def\printtitle{%						% Define command: \printtitle
    {\color{bl} \centering \huge \sc \textbf{\@title}\par}}		% Typesetting
\makeatother							% End definition

\title{Distribuciones de probabilidad \\ 
		\large \vspace*{-10pt} para el an\'alisis de Supervivencia\vspace*{10pt}}

% Define a new command that prints the author(s) only
\makeatletter							% Begin definition
\def\printauthor{%					% Define command: \printauthor
    {\centering \small \@author}}				% Typesetting
\makeatother							% End definition

\author{%
	Karen Viviana Rodr\'iguez Mac\'ias \\
	14 de Octubre del 2016 \\
	\vspace{20pt}
	}

% Custom headers and footers
\usepackage{fancyhdr}
	\pagestyle{fancy}					% Enabling the custom headers/footers
\usepackage{lastpage}	
	% Header (empty)
	\lhead{}
	\chead{}
	\rhead{}
	% Footer (you may change this to your own needs)
	\lfoot{\footnotesize Estad\'istica III}
	\cfoot{}
	\rfoot{\footnotesize Página \thepage\ de \pageref{LastPage}}	% "Page 1 of 2"
	\renewcommand{\headrulewidth}{0.0pt}
	\renewcommand{\footrulewidth}{0.4pt}

% Change the abstract environment
\usepackage[runin]{abstract}			% runin option for a run-in title
\setlength\absleftindent{30pt}		% left margin
\setlength\absrightindent{30pt}		% right margin
\abslabeldelim{\quad}						% 
\setlength{\abstitleskip}{-10pt}
\renewcommand{\abstractname}{}
\renewcommand{\abstracttextfont}{\color{bl} \small \slshape}	% slanted text


%%% Start of the document
\begin{document}
%%% Top of the page: Author, Title and Abstact
\printtitle 

\printauthor

\begin{abstract}
En el presente documento se definen 5 distribuciones que se pueden utilizar para el análisis de supervivencia, ya que tienen soporte en los positivos. Se obtiene su función de distribución, de supervivencia y de riesgo utilizando formulas de equivalencia entre ellas.
\end{abstract}

%%% Start of the 'real' content of the article, using a two column layout
\section{Distribuci\'on Pareto}
Sea $ x ~ Pareto(\lambda,\kappa) $ decimos que $x$ se distribuye Pareto con parametros $\lambda$ y $\kappa$. La variable aleatoria Pareto $x$ con parámetros positivos $\lambda$ y $\kappa$ tiene la siguiente función de distribución acumulativa :

\begin{equation*}
	F(x) = P(X \leqslant x) = 1 -  \left( \frac{\lambda}{x} \right) ^ \kappa \hspace{1.5cm} x>\lambda
\end{equation*}

\subsubsection*{Funci\'on de densidad}

\begin{equation*}
\boxed{
f(x) = \frac{\kappa \lambda ^ \kappa}{ x^{\kappa+1}} \hspace{1.5cm} x > \lambda}
\end{equation*}
Esto es porque dada la función de distribución acumulativa podemos obtener la función de densidad con su derivada, como se muestra a continuación : 
\begin{align*}
f(x) &= \frac{\partial F(x)}{\partial x}	 \\
	&=	\frac{\partial}{\partial x} \left(1 - \frac{\lambda ^ \kappa}{x ^ \kappa} \right)\\
	&= (-\lambda ^ \kappa) \left( \frac{-\kappa x^{\kappa-1}}{x^{2\kappa}} \right) \\
	&= (-\lambda ^ \kappa) \left( \frac{-\kappa}{x^{\kappa+1}} \right)\\
	&= \frac{\kappa \lambda^\kappa}{x^{\kappa+1}}
\end{align*} 

\subsubsection*{Funci\'on de Supervivencia}
\begin{equation*}
\boxed{
S(x) =\left( \frac{\lambda}{x} \right) ^ \kappa \hspace{1.5cm} x > \lambda}
\end{equation*}
\newpage
La función de Supervivencia en el soporte de x se obtiene de la siguiente manera.

\begin{align*}
S(x) &= P(X \geqslant x) \\
     &= 1 - F(x) \\
     &= 1 - \left(1-\frac{\lambda}{x} \right)^\kappa  \\
     &=\left( \frac{\lambda}{x} \right) ^ \kappa
\end{align*}

\subsubsection{Funci\'on de riesgo}
\begin{equation*}
\boxed{
h(x) = \frac{\kappa}{x}  \hspace{1.5cm} x > \lambda}
\end{equation*}

La función de riesgo en el soporte de x se obtiene de la siguiente manera.

\begin{align*}
h(x) &= \frac{f(x)}{S(x)} \\
     &= \frac{\frac{\kappa \lambda ^ \kappa}{ x^{\kappa+1}}}{\left( \frac{\lambda}{x} \right) ^ \kappa} \\
     &= \frac{x^\kappa \kappa \lambda^\kappa}{x^{\kappa+1} \lambda^\kappa}   \\
     &= \frac{\kappa}{x}
\end{align*}

\subsubsection{Funci\'on de riesgo acumulativa}
\begin{equation*}
\boxed{
H(x) = -\kappa (\ln \lambda - \ln x)  \hspace{1.5cm} x > \lambda}
\end{equation*}
La función de riesgo acumulativa en el soporte de x se obtiene de la siguiente manera.

\begin{align*}
H(x) &= -\ln S(x) \\
     &= - \ln \left( \frac{\lambda}{x} \right) ^ \kappa \\
     &= -\kappa (\ln \lambda - \ln x)
\end{align*}

\section{Distribuci\'on Muth}

\section{Distribuci\'on Lomax}

\section{Distribuci\'on Rayleigh}

\section{Distribuci\'on Poder (Power Distribution)}


\end{document}
